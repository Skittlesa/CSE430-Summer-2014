\documentclass[12pt]{article}
\usepackage{geometry}                % See geometry.pdf to learn the layout options. There are lots.
\geometry{letterpaper}                   % ... or a4paper or a5paper or ... 
\usepackage{graphicx}
\usepackage{color}
\usepackage{amssymb}
\usepackage{epstopdf}
\usepackage{verbatim}
\DeclareGraphicsRule{.tif}{png}{.png}{`convert #1 `dirname #1`/`basename #1 .tif`.png}

\title{CSE 430 Summer 2014 - Project 2}
\author{Ryan Dougherty}
\date{}                                           % Activate to display a given date or no date

\begin{document}
\maketitle

\abstract{This document will cover Project 2, which is about solutions to the Reader-Writer problem. We will observe the output produced by the code given, then modify the critical section in Database.java to implement the starvation-free pseudo-code algorithm (but is unfair), and finally modify RWQueue so that our implementation is fair.}

\section{Part 1 Output}
The Part 1 output is the following:
{\color{blue}
\verbatiminput{part1.txt}
}
As we can see, there are some sections that allow reader and writer 0 to take much more time than 1 or 2. Therefore, this algorithm is not starvation-free. 

\section{Part 2 Output}

\section{Part 3 Output}
The purpose of these programs: to create a server-client relationship in Java to provide a service to a client. The functionality is to give the current date and time. The program output is the following:
{\color{blue}
\newline Thu May 29 10:11:16 MST 2014
}

\end{document}  
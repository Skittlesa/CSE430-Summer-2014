\documentclass[12pt]{article}
\usepackage{geometry}                % See geometry.pdf to learn the layout options. There are lots.
\geometry{letterpaper}                   % ... or a4paper or a5paper or ... 
\usepackage{graphicx}
\usepackage{color}
\usepackage{amssymb}
\usepackage{epstopdf}
\DeclareGraphicsRule{.tif}{png}{.png}{`convert #1 `dirname #1`/`basename #1 .tif`.png}

\title{CSE 430 Homework 7}
\author{Ryan Dougherty}
\date{}                                           % Activate to display a given date or no date

\begin{document}
\maketitle

\section*{Question 10.6}{\color{blue}Is there any way to implement truly stable storage? Explain your answer.} It is not possible to implement truly stable storage in totality, but in practice it is possible. What is to be done is to be able to manage copies of the data; if a copy is deleted for any reason, a different version still exists. But it is possible for all copies to be deleted, so that the data cannot be recovered.

\section*{Question 10.11}{\color{blue}Suppose that a disk drive has 5,000 cylinders, numbered 0 to 4,999. The drive is currently serving a request at cylinder 2,150, and the previous request was at cylinder 1,805. The queue of pending requests, in FIFO order, is:
\begin{center}
2,069, 1,212, 2,296, 2,800, 544, 1,618, 356, 1,523, 4,965, 3,681
\end{center}
Starting from the current head position, what is the total distance (in cylinders) that the disk arm moves to satisfy all the pending requests for each of the following disk-scheduling algorithms?
\begin{enumerate}
\item[(a)]FCFS {\color{black}The schedule is 2150, 2069, 1212, 2296, 2800, 544, 1618, 356, 1523, 4965, 3681. The total distance is 13011.
}
\item[(b)]SSTF {\color{black}The schedule is 2150, 2069, 2296, 2800, 3681, 4965, 1618, 1523, 1212, 544, 356. The total distance is 7586.
}
\item[(c)]SCAN {\color{black}The schedule is 2150, 2069, 1618, 1523, 1212, 544, 356, 0, 2296, 2800, 3681, 4965. The total distance is 7115.
}
\item[(d)]LOOK {\color{black}The schedule is 2150, 2069, 1618, 1523, 1212, 544, 356, 2296, 2800, 3681, 4965. The total distance is 6403.
}
\item[(e)]C-SCAN {\color{black}The schedule is 2150, 2296, 2800, 3681, 4965, 4999, 0, 356, 544, 1212, 1523, 1618, 2069. The total distance is 9917.
}
\item[(f)]C-LOOK {\color{black}The schedule is 2150, 2296, 2800, 3681, 4965, 356, 544, 1212, 1523, 1618, 2069. The total distance is 9137.
}
\end{enumerate}
}

\section*{Question 10.14}{\color{blue}Describe some advantages and disadvantages of using SSDs as a caching tier and as a disk-drive replacement compared with using only magnetic disks.} An advantage of using SSDs is that they are faster than magnetic disks because, unlike magnetic disks, there are no moving parts (i.e. no seek time nor rotational latency).
\\ \\
A disadvantage of using SSDs is that they are much more expensive than magnetic disks, and have less capacity. So for systems with large caching tiers, SSDs are not appropriate.

\section*{Question 10.17}{\color{blue}Consider a RAID level 5 organization comprising five disks, with the parity for sets of four blocks on four disks stored on the fifth disk. How many blocks are accessed in order to perform the following?
\begin{enumerate}
\item[(a)]A write of one block of data. {\color{black}
The following is an algorithm to access one block of data:
\begin{enumerate}
\item[(1)]Parity block is read.
\item[(2)]Data originally from needed block is read.
\item[(3)]Compute new parity block depending on needed block's contents (both before and after).
\item[(4)]Write both blocks.
\end{enumerate}
Therefore, there are 1+1+2 = 4 blocks accessed.
}
\item[(b)]A write of seven continous blocks of data {\color{black}Assume that these blocks start with a boundary of 4 blocks. The algorithm to read the 7 blocks is:
\begin{enumerate}
\item[(1)]Write these seven blocks.
\item[(2)]Write the parity block of the first 4.
\item[(3)]Read the 8th block.
\item[(4)]Compute parity block for the next 4.
\item[(5)]Write corresponding parity block.
\end{enumerate}
Therefore, there are 7+1+1+1 = 10 blocks accessed.
}
\end{enumerate}

\end{document}  
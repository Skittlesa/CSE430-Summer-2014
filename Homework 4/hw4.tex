\documentclass[12pt]{article}
\usepackage{geometry}                % See geometry.pdf to learn the layout options. There are lots.
\geometry{letterpaper}                   % ... or a4paper or a5paper or ... 
\usepackage{graphicx}
\usepackage{color}
\usepackage{amssymb}
\usepackage{epstopdf}
\DeclareGraphicsRule{.tif}{png}{.png}{`convert #1 `dirname #1`/`basename #1 .tif`.png}

\title{CSE 430 Homework 4}
\author{Ryan Dougherty}
\date{}                                           % Activate to display a given date or no date

\begin{document}
\maketitle

\section*{Question 6.16} {\color{blue}Consider the following set of processes, with the length of the CPU burst...The processes are assumed to have arrieved in the order $P_{1}, P_{2}, P_{3}, P_{4}, P_{5}$, all at time 0.
\begin{enumerate}
\item[(a)]Draw 4 Gantt charts that illustrate the execution of these processes using the following scheduling algorithms: FCFS, SJF, nonpreemptive priority (a larger priority number implies a higher priority), and RR (quantum = 2).
\\
{\color{black}
FCFS: \\
\framebox{$P_{1}$}\framebox{$P_{2}$}\framebox{$P_{3}$}\framebox{$P_{4}$}\framebox{$P_{5}$}
\begin{verbatim}
0  2  3 11 15 20
\end{verbatim}

SJF: \\
\framebox{$P_{2}$}\framebox{$P_{1}$}\framebox{$P_{4}$}\framebox{$P_{5}$}\framebox{$P_{3}$}
\begin{verbatim}
0  1  3  7 12 20
\end{verbatim}

Nonpreemptive Priority: \\
\framebox{$P_{3}$}\framebox{$P_{5}$}\framebox{$P_{1}$}\framebox{$P_{4}$}\framebox{$P_{2}$}
\begin{verbatim}
0  8 13 15 19 20
\end{verbatim}

RR: \\
\framebox{$P_{1}$}\framebox{$P_{2}$}\framebox{$P_{3}$}\framebox{$P_{4}$}\framebox{$P_{5}$}\framebox{$P_{3}$}\framebox{$P_{4}$}\framebox{$P_{5}$}\framebox{$P_{3}$}\framebox{$P_{5}$}\framebox{$P_{3}$}
\begin{verbatim}
0  2  3  5  7  9  11 13 15 17 18 20
\end{verbatim}

}

\newpage
\item[(b)]What is the turnaround time of each process for each of the scheduling algorithms in part a?
\\
{\color{black}
\begin{table}[h]
\begin{tabular}{lllll}
Process & FCFS & SJF & Nonpreemptive Priority & RR \\
$P_{1}$      & 2    & 3  & 15       & 2 \\
$P_{2}$      & 3    & 1   & 20       & 3  \\
$P_{3}$      & 11   & 20  & 8        & 20 \\
$P_{4}$      & 15   & 7   & 19       & 13  \\
$P_{5}$      & 20   & 12   & 13       & 18 \\
Average & 10.2  & 8.6   & 15      &   10.3 
\end{tabular}
\end{table}
}

\item[(c)]What is the waiting time of each process for each of these scheduling algorithms?
\\
{\color{black}
\begin{table}[h1]
\begin{tabular}{lllll}
Process & FCFS & SJF & Nonpreemptive Priority & RR   \\
$P_{1}$      & 0    & 1   & 13        & 0    \\
$P_{2}$      & 2    & 0   & 19        & 2    \\
$P_{3}$      & 3    & 12  & 0        & 12   \\
$P_{4}$      & 11   & 3   & 15       & 9    \\
$P_{5}$      & 15   & 7   & 8       & 13    \\
Average & 6.2  & 4.6   & 11      & 7.2
\end{tabular}
\end{table}
}

\item[(d)]Which of the algorithms results in the minimum average waiting time (over all processes)? {\color{black}As seen in the tables above, the minimum average waiting time is the shortest-job-first (SJF) algorithm. }
\end{enumerate}
}

\section*{Question 6.19} {\color{blue}Which of the following scheduling algorithms could result in starvation?
\begin{enumerate}
\item[(a)]First-come, first-served {\color{black} No, it is not possible.}
\item[(b)]Shortest job first {\color{black} Yes, it is possible.}
\item[(c)]Round robin {\color{black} No, it is not possible.}
\item[(d)]Priority {\color{black} Yes, it is possible.}
\end{enumerate}
}

\section*{Question 6.20} {\color{blue}Consider a variant of the RR scheduling algorithm in which the entries in the ready queue are pointers to the PCBs.
\begin{enumerate}
\item[(a)]What would be the effect of putting two pointers to the same process in the ready queue? {\color{black} That process will have a higher priority (i.e. run twice as often). }
\item[(b)]What would be two major advantages and two disadvantages of this scheme?
{\color{black}
\begin{itemize}
\item Advantages
\begin{enumerate}
\item[(i)]The user does not have to change the scheduling algorithm.
\item[(ii)]For processes of lower priority, the algoritm prevents starvation.
\end{enumerate}
\item Disadvantages
\begin{enumerate}
\item[(i)]The user must search through the entire list of processes in order to remove a process.
\item[(ii)]There is more time taken in context switches.
\end{enumerate}
\end{itemize}
}
\item[(c)]How would you modify the basic RR algorithm to achieve the same effect without the duplicate pointers? {\color{black} Achieving the same effect without duplicate pointers can happen if we allow the quantum time of each process to be changed on a per-process basis.}
\end{enumerate}
}

\section*{Question 6.24} {\color{blue}Explain the differences in how much the following scheduling algorithms discriminate in favor of short processes:
\begin{enumerate}
\item[(a)]FCFS {\color{black} If long processes come before short processes, then the short processes will have to wait longer before they can run. Therefore, FCFS discriminates against shorter processes.}
\item[(b)]RR {\color{black} Since each process has equal priority, the shorter processes will finish first. Therefore, RR discriminates for shorter processes. }
\item[(c)]Multilevel feedback queues {\color{black} If a process take too much CPU time, it is moved to a lower-priority queue (i.e. longer processes). Therefore, multilevel feedback queues discriminate for shorter processes.}
\end{enumerate}
}
\end{document}  
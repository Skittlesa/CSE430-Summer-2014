\documentclass[12pt]{article}
\usepackage{geometry}                % See geometry.pdf to learn the layout options. There are lots.
\geometry{letterpaper}                   % ... or a4paper or a5paper or ... 
\usepackage{graphicx}
\usepackage{color}
\usepackage{amssymb}
\usepackage{epstopdf}
\DeclareGraphicsRule{.tif}{png}{.png}{`convert #1 `dirname #1`/`basename #1 .tif`.png}

\title{CSE 430 Summer 2014 - Homework 1}
\author{Ryan Dougherty}
\date{}                                           % Activate to display a given date or no date

\begin{document}
\maketitle

\section*{Question 1.13} {\color{blue}The issue of resource utilization shows up in different forms in different types of operating systems. List what resources must be managed carefully in the following settings: (a) Mainframe or minicomputer systems, (b) Workstations connected to servers, (c) Mobile computers}
(a) (b) (c)

\section*{Question 1.22} {\color{blue}Many SMP systems have different levels of caches; one level is local to each processing core, and another level is shared among all processing cores. Why are caching systems designed this way?}

\section*{Question 1.23} {\color{blue}Consider an SMP system similar to the one shown in Figure 1.6. Illustrate with an example how data residing in memory could in fact have a different value in each of the local caches.}

\section*{Question 1.25} {\color{blue}Describe a mechanism for enforcing memory protection in order to prevent a program from modifying the memory associated with other programs.}

\section*{Question 2.5} {\color{blue}What is the purpose of the command interpreter? Why is it usually separate from the kernel?}

\section*{Question 2.17} {\color{blue}Would it be possible for the user to develop a new command interpreter using the system-call interface provided by the operating system?}

\section*{Question 2.18} {\color{blue}What are the two models of interprocess communication? What are the strengths and weaknesses of the two approaches?}

\end{document}  